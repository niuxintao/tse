
%% bare_jrnl_compsoc.tex
%% V1.3
%% 2007/01/11
%% by Michael Shell
%% See:
%% http://www.michaelshell.org/
%% for current contact information.
%%
%% This is a skeleton file demonstrating the use of IEEEtran.cls
%% (requires IEEEtran.cls version 1.7 or later) with an IEEE Computer
%% Society journal paper.
%%
%% Support sites:
%% http://www.michaelshell.org/tex/ieeetran/
%% http://www.ctan.org/tex-archive/macros/latex/contrib/IEEEtran/
%% and
%% http://www.ieee.org/

%%*************************************************************************
%% Legal Notice:
%% This code is offered as-is without any warranty either expressed or
%% implied; without even the implied warranty of MERCHANTABILITY or
%% FITNESS FOR A PARTICULAR PURPOSE!
%% User assumes all risk.
%% In no event shall IEEE or any contributor to this code be liable for
%% any damages or losses, including, but not limited to, incidental,
%% consequential, or any other damages, resulting from the use or misuse
%% of any information contained here.
%%
%% All comments are the opinions of their respective authors and are not
%% necessarily endorsed by the IEEE.
%%
%% This work is distributed under the LaTeX Project Public License (LPPL)
%% ( http://www.latex-project.org/ ) version 1.3, and may be freely used,
%% distributed and modified. A copy of the LPPL, version 1.3, is included
%% in the base LaTeX documentation of all distributions of LaTeX released
%% 2003/12/01 or later.
%% Retain all contribution notices and credits.
%% ** Modified files should be clearly indicated as such, including  **
%% ** renaming them and changing author support contact information. **
%%
%% File list of work: IEEEtran.cls, IEEEtran_HOWTO.pdf, bare_adv.tex,
%%                    bare_conf.tex, bare_jrnl.tex, bare_jrnl_compsoc.tex
%%*************************************************************************

% *** Authors should verify (and, if needed, correct) their LaTeX system  ***
% *** with the testflow diagnostic prior to trusting their LaTeX platform ***
% *** with production work. IEEE's font choices can trigger bugs that do  ***
% *** not appear when using other class files.                            ***
% The testflow support page is at:
% http://www.michaelshell.org/tex/testflow/




% Note that the a4paper option is mainly intended so that authors in
% countries using A4 can easily print to A4 and see how their papers will
% look in print - the typesetting of the document will not typically be
% affected with changes in paper size (but the bottom and side margins will).
% Use the testflow package mentioned above to verify correct handling of
% both paper sizes by the user's LaTeX system.
%
% Also note that the "draftcls" or "draftclsnofoot", not "draft", option
% should be used if it is desired that the figures are to be displayed in
% draft mode.
%
% The Computer Society usually requires 10pt for submissions.
%

\documentclass[10pt,journal,cspaper,compsoc]{IEEEtran}
\usepackage{graphicx}
\usepackage{amsthm}
\usepackage{algpseudocode}
\usepackage{algorithm}
\usepackage{amsmath}

\algrenewcommand{\algorithmicrequire}{\textbf{Input:}}
\algrenewcommand{\algorithmicensure}{\textbf{Output:}}
\renewcommand{\algorithmicforall}{\textbf{for each}}
%
% If IEEEtran.cls has not been installed into the LaTeX system files,
% manually specify the path to it like:
% \documentclass[12pt,journal,compsoc]{../sty/IEEEtran}





% Some very useful LaTeX packages include:
% (uncomment the ones you want to load)


% *** MISC UTILITY PACKAGES ***
%
%\usepackage{ifpdf}
% Heiko Oberdiek's ifpdf.sty is very useful if you need conditional
% compilation based on whether the output is pdf or dvi.
% usage:
% \ifpdf
%   % pdf code
% \else
%   % dvi code
% \fi
% The latest version of ifpdf.sty can be obtained from:
% http://www.ctan.org/tex-archive/macros/latex/contrib/oberdiek/
% Also, note that IEEEtran.cls V1.7 and later provides a builtin
% \ifCLASSINFOpdf conditional that works the same way.
% When switching from latex to pdflatex and vice-versa, the compiler may
% have to be run twice to clear warning/error messages.






% *** CITATION PACKAGES ***
%
\ifCLASSOPTIONcompsoc
  % IEEE Computer Society needs nocompress option
  % requires cite.sty v4.0 or later (November 2003)
  % \usepackage[nocompress]{cite}
\else
  % normal IEEE
  % \usepackage{cite}
\fi
% cite.sty was written by Donald Arseneau
% V1.6 and later of IEEEtran pre-defines the format of the cite.sty package
% \cite{} output to follow that of IEEE. Loading the cite package will
% result in citation numbers being automatically sorted and properly
% "compressed/ranged". e.g., [1], [9], [2], [7], [5], [6] without using
% cite.sty will become [1], [2], [5]--[7], [9] using cite.sty. cite.sty's
% \cite will automatically add leading space, if needed. Use cite.sty's
% noadjust option (cite.sty V3.8 and later) if you want to turn this off.
% cite.sty is already installed on most LaTeX systems. Be sure and use
% version 4.0 (2003-05-27) and later if using hyperref.sty. cite.sty does
% not currently provide for hyperlinked citations.
% The latest version can be obtained at:
% http://www.ctan.org/tex-archive/macros/latex/contrib/cite/
% The documentation is contained in the cite.sty file itself.
%
% Note that some packages require special options to format as the Computer
% Society requires. In particular, Computer Society  papers do not use
% compressed citation ranges as is done in typical IEEE papers
% (e.g., [1]-[4]). Instead, they list every citation separately in order
% (e.g., [1], [2], [3], [4]). To get the latter we need to load the cite
% package with the nocompress option which is supported by cite.sty v4.0
% and later. Note also the use of a CLASSOPTION conditional provided by
% IEEEtran.cls V1.7 and later.





% *** GRAPHICS RELATED PACKAGES ***
%
\ifCLASSINFOpdf
  % \usepackage[pdftex]{graphicx}
  % declare the path(s) where your graphic files are
  % \graphicspath{{../pdf/}{../jpeg/}}
  % and their extensions so you won't have to specify these with
  % every instance of \includegraphics
  % \DeclareGraphicsExtensions{.pdf,.jpeg,.png}
\else
  % or other class option (dvipsone, dvipdf, if not using dvips). graphicx
  % will default to the driver specified in the system graphics.cfg if no
  % driver is specified.
  % \usepackage[dvips]{graphicx}
  % declare the path(s) where your graphic files are
  % \graphicspath{{../eps/}}
  % and their extensions so you won't have to specify these with
  % every instance of \includegraphics
  % \DeclareGraphicsExtensions{.eps}
\fi
% graphicx was written by David Carlisle and Sebastian Rahtz. It is
% required if you want graphics, photos, etc. graphicx.sty is already
% installed on most LaTeX systems. The latest version and documentation can
% be obtained at:
% http://www.ctan.org/tex-archive/macros/latex/required/graphics/
% Another good source of documentation is "Using Imported Graphics in
% LaTeX2e" by Keith Reckdahl which can be found as epslatex.ps or
% epslatex.pdf at: http://www.ctan.org/tex-archive/info/
%
% latex, and pdflatex in dvi mode, support graphics in encapsulated
% postscript (.eps) format. pdflatex in pdf mode supports graphics
% in .pdf, .jpeg, .png and .mps (metapost) formats. Users should ensure
% that all non-photo figures use a vector format (.eps, .pdf, .mps) and
% not a bitmapped formats (.jpeg, .png). IEEE frowns on bitmapped formats
% which can result in "jaggedy"/blurry rendering of lines and letters as
% well as large increases in file sizes.
%
% You can find documentation about the pdfTeX application at:
% http://www.tug.org/applications/pdftex





% *** MATH PACKAGES ***
%
%\usepackage[cmex10]{amsmath}
% A popular package from the American Mathematical Society that provides
% many useful and powerful commands for dealing with mathematics. If using
% it, be sure to load this package with the cmex10 option to ensure that
% only type 1 fonts will utilized at all point sizes. Without this option,
% it is possible that some math symbols, particularly those within
% footnotes, will be rendered in bitmap form which will result in a
% document that can not be IEEE Xplore compliant!
%
% Also, note that the amsmath package sets \interdisplaylinepenalty to 10000
% thus preventing page breaks from occurring within multiline equations. Use:
%\interdisplaylinepenalty=2500
% after loading amsmath to restore such page breaks as IEEEtran.cls normally
% does. amsmath.sty is already installed on most LaTeX systems. The latest
% version and documentation can be obtained at:
% http://www.ctan.org/tex-archive/macros/latex/required/amslatex/math/





% *** SPECIALIZED LIST PACKAGES ***
%
%\usepackage{algorithmic}
% algorithmic.sty was written by Peter Williams and Rogerio Brito.
% This package provides an algorithmic environment fo describing algorithms.
% You can use the algorithmic environment in-text or within a figure
% environment to provide for a floating algorithm. Do NOT use the algorithm
% floating environment provided by algorithm.sty (by the same authors) or
% algorithm2e.sty (by Christophe Fiorio) as IEEE does not use dedicated
% algorithm float types and packages that provide these will not provide
% correct IEEE style captions. The latest version and documentation of
% algorithmic.sty can be obtained at:
% http://www.ctan.org/tex-archive/macros/latex/contrib/algorithms/
% There is also a support site at:
% http://algorithms.berlios.de/index.html
% Also of interest may be the (relatively newer and more customizable)
% algorithmicx.sty package by Szasz Janos:
% http://www.ctan.org/tex-archive/macros/latex/contrib/algorithmicx/




% *** ALIGNMENT PACKAGES ***
%
%\usepackage{array}
% Frank Mittelbach's and David Carlisle's array.sty patches and improves
% the standard LaTeX2e array and tabular environments to provide better
% appearance and additional user controls. As the default LaTeX2e table
% generation code is lacking to the point of almost being broken with
% respect to the quality of the end results, all users are strongly
% advised to use an enhanced (at the very least that provided by array.sty)
% set of table tools. array.sty is already installed on most systems. The
% latest version and documentation can be obtained at:
% http://www.ctan.org/tex-archive/macros/latex/required/tools/


%\usepackage{mdwmath}
%\usepackage{mdwtab}
% Also highly recommended is Mark Wooding's extremely powerful MDW tools,
% especially mdwmath.sty and mdwtab.sty which are used to format equations
% and tables, respectively. The MDWtools set is already installed on most
% LaTeX systems. The lastest version and documentation is available at:
% http://www.ctan.org/tex-archive/macros/latex/contrib/mdwtools/


% IEEEtran contains the IEEEeqnarray family of commands that can be used to
% generate multiline equations as well as matrices, tables, etc., of high
% quality.


%\usepackage{eqparbox}
% Also of notable interest is Scott Pakin's eqparbox package for creating
% (automatically sized) equal width boxes - aka "natural width parboxes".
% Available at:
% http://www.ctan.org/tex-archive/macros/latex/contrib/eqparbox/





% *** SUBFIGURE PACKAGES ***
%\ifCLASSOPTIONcompsoc
%\usepackage[tight,normalsize,sf,SF]{subfigure}
%\else
%\usepackage[tight,footnotesize]{subfigure}
%\fi
% subfigure.sty was written by Steven Douglas Cochran. This package makes it
% easy to put subfigures in your figures. e.g., "Figure 1a and 1b". For IEEE
% work, it is a good idea to load it with the tight package option to reduce
% the amount of white space around the subfigures. Computer Society papers
% use a larger font and \sffamily font for their captions, hence the
% additional options needed under compsoc mode. subfigure.sty is already
% installed on most LaTeX systems. The latest version and documentation can
% be obtained at:
% http://www.ctan.org/tex-archive/obsolete/macros/latex/contrib/subfigure/
% subfigure.sty has been superceeded by subfig.sty.


%\ifCLASSOPTIONcompsoc
%  \usepackage[caption=false]{caption}
%  \usepackage[font=normalsize,labelfont=sf,textfont=sf]{subfig}
%\else
%  \usepackage[caption=false]{caption}
%  \usepackage[font=footnotesize]{subfig}
%\fi
% subfig.sty, also written by Steven Douglas Cochran, is the modern
% replacement for subfigure.sty. However, subfig.sty requires and
% automatically loads Axel Sommerfeldt's caption.sty which will override
% IEEEtran.cls handling of captions and this will result in nonIEEE style
% figure/table captions. To prevent this problem, be sure and preload
% caption.sty with its "caption=false" package option. This is will preserve
% IEEEtran.cls handing of captions. Version 1.3 (2005/06/28) and later
% (recommended due to many improvements over 1.2) of subfig.sty supports
% the caption=false option directly:
%\ifCLASSOPTIONcompsoc
%  \usepackage[caption=false,font=normalsize,labelfont=sf,textfont=sf]{subfig}
%\else
%  \usepackage[caption=false,font=footnotesize]{subfig}
%\fi
%
% The latest version and documentation can be obtained at:
% http://www.ctan.org/tex-archive/macros/latex/contrib/subfig/
% The latest version and documentation of caption.sty can be obtained at:
% http://www.ctan.org/tex-archive/macros/latex/contrib/caption/




% *** FLOAT PACKAGES ***
%
%\usepackage{fixltx2e}
% fixltx2e, the successor to the earlier fix2col.sty, was written by
% Frank Mittelbach and David Carlisle. This package corrects a few problems
% in the LaTeX2e kernel, the most notable of which is that in current
% LaTeX2e releases, the ordering of single and double column floats is not
% guaranteed to be preserved. Thus, an unpatched LaTeX2e can allow a
% single column figure to be placed prior to an earlier double column
% figure. The latest version and documentation can be found at:
% http://www.ctan.org/tex-archive/macros/latex/base/



%\usepackage{stfloats}
% stfloats.sty was written by Sigitas Tolusis. This package gives LaTeX2e
% the ability to do double column floats at the bottom of the page as well
% as the top. (e.g., "\begin{figure*}[!b]" is not normally possible in
% LaTeX2e). It also provides a command:
%\fnbelowfloat
% to enable the placement of footnotes below bottom floats (the standard
% LaTeX2e kernel puts them above bottom floats). This is an invasive package
% which rewrites many portions of the LaTeX2e float routines. It may not work
% with other packages that modify the LaTeX2e float routines. The latest
% version and documentation can be obtained at:
% http://www.ctan.org/tex-archive/macros/latex/contrib/sttools/
% Documentation is contained in the stfloats.sty comments as well as in the
% presfull.pdf file. Do not use the stfloats baselinefloat ability as IEEE
% does not allow \baselineskip to stretch. Authors submitting work to the
% IEEE should note that IEEE rarely uses double column equations and
% that authors should try to avoid such use. Do not be tempted to use the
% cuted.sty or midfloat.sty packages (also by Sigitas Tolusis) as IEEE does
% not format its papers in such ways.




%\ifCLASSOPTIONcaptionsoff
%  \usepackage[nomarkers]{endfloat}
% \let\MYoriglatexcaption\caption
% \renewcommand{\caption}[2][\relax]{\MYoriglatexcaption[#2]{#2}}
%\fi
% endfloat.sty was written by James Darrell McCauley and Jeff Goldberg.
% This package may be useful when used in conjunction with IEEEtran.cls'
% captionsoff option. Some IEEE journals/societies require that submissions
% have lists of figures/tables at the end of the paper and that
% figures/tables without any captions are placed on a page by themselves at
% the end of the document. If needed, the draftcls IEEEtran class option or
% \CLASSINPUTbaselinestretch interface can be used to increase the line
% spacing as well. Be sure and use the nomarkers option of endfloat to
% prevent endfloat from "marking" where the figures would have been placed
% in the text. The two hack lines of code above are a slight modification of
% that suggested by in the endfloat docs (section 8.3.1) to ensure that
% the full captions always appear in the list of figures/tables - even if
% the user used the short optional argument of \caption[]{}.
% IEEE papers do not typically make use of \caption[]'s optional argument,
% so this should not be an issue. A similar trick can be used to disable
% captions of packages such as subfig.sty that lack options to turn off
% the subcaptions:
% For subfig.sty:
% \let\MYorigsubfloat\subfloat
% \renewcommand{\subfloat}[2][\relax]{\MYorigsubfloat[]{#2}}
% For subfigure.sty:
% \let\MYorigsubfigure\subfigure
% \renewcommand{\subfigure}[2][\relax]{\MYorigsubfigure[]{#2}}
% However, the above trick will not work if both optional arguments of
% the \subfloat/subfig command are used. Furthermore, there needs to be a
% description of each subfigure *somewhere* and endfloat does not add
% subfigure captions to its list of figures. Thus, the best approach is to
% avoid the use of subfigure captions (many IEEE journals avoid them anyway)
% and instead reference/explain all the subfigures within the main caption.
% The latest version of endfloat.sty and its documentation can obtained at:
% http://www.ctan.org/tex-archive/macros/latex/contrib/endfloat/
%
% The IEEEtran \ifCLASSOPTIONcaptionsoff conditional can also be used
% later in the document, say, to conditionally put the References on a
% page by themselves.




% *** PDF, URL AND HYPERLINK PACKAGES ***
%
%\usepackage{url}
% url.sty was written by Donald Arseneau. It provides better support for
% handling and breaking URLs. url.sty is already installed on most LaTeX
% systems. The latest version can be obtained at:
% http://www.ctan.org/tex-archive/macros/latex/contrib/misc/
% Read the url.sty source comments for usage information. Basically,
% \url{my_url_here}.





% *** Do not adjust lengths that control margins, column widths, etc. ***
% *** Do not use packages that alter fonts (such as pslatex).         ***
% There should be no need to do such things with IEEEtran.cls V1.6 and later.
% (Unless specifically asked to do so by the journal or conference you plan
% to submit to, of course. )


% correct bad hyphenation here
\hyphenation{op-tical net-works semi-conduc-tor}


\begin{document}
%
% paper title
% can use linebreaks \\ within to get better formatting as desired
\title{Efficiently Identify the failure-inducing schemas}
%
%
% author names and IEEE memberships
% note positions of commas and nonbreaking spaces ( ~ ) LaTeX will not break
% a structure at a ~ so this keeps an author's name from being broken across
% two lines.
% use \thanks{} to gain access to the first footnote area
% a separate \thanks must be used for each paragraph as LaTeX2e's \thanks
% was not built to handle multiple paragraphs
%
%
%\IEEEcompsocitemizethanks is a special \thanks that produces the bulleted
% lists the Computer Society journals use for "first footnote" author
% affiliations. Use \IEEEcompsocthanksitem which works much like \item
% for each affiliation group. When not in compsoc mode,
% \IEEEcompsocitemizethanks becomes like \thanks and
% \IEEEcompsocthanksitem becomes a line break with idention. This
% facilitates dual compilation, although admittedly the differences in the
% desired content of \author between the different types of papers makes a
% one-size-fits-all approach a daunting prospect. For instance, compsoc
% journal papers have the author affiliations above the "Manuscript
% received ..."  text while in non-compsoc journals this is reversed. Sigh.

\author{Xintao~Niu,~
        Changhai~Nie,~\IEEEmembership{Member,~IEEE,}
        and~~% <-this % stops a space
\IEEEcompsocitemizethanks{\IEEEcompsocthanksitem M. Shell is with the Department
of Electrical and Computer Engineering, Georgia Institute of Technology, Atlanta,
GA, 30332.\protect\\
% note need leading \protect in front of \\ to get a newline within \thanks as
% \\ is fragile and will error, could use \hfil\break instead.
E-mail: see http://www.michaelshell.org/contact.html
\IEEEcompsocthanksitem J. Doe and J. Doe are with Anonymous University.}% <-this % stops a space
\thanks{}}

% note the % following the last \IEEEmembership and also \thanks -
% these prevent an unwanted space from occurring between the last author name
% and the end of the author line. i.e., if you had this:
%
% \author{....lastname \thanks{...} \thanks{...} }
%                     ^------------^------------^----Do not want these spaces!
%
% a space would be appended to the last name and could cause every name on that
% line to be shifted left slightly. This is one of those "LaTeX things". For
% instance, "\textbf{A} \textbf{B}" will typeset as "A B" not "AB". To get
% "AB" then you have to do: "\textbf{A}\textbf{B}"
% \thanks is no different in this regard, so shield the last } of each \thanks
% that ends a line with a % and do not let a space in before the next \thanks.
% Spaces after \IEEEmembership other than the last one are OK (and needed) as
% you are supposed to have spaces between the names. For what it is worth,
% this is a minor point as most people would not even notice if the said evil
% space somehow managed to creep in.



% The paper headers
\markboth{Journal of \LaTeX\ Class Files,~Vol.~6, No.~1, January~2007}%
{Shell \MakeLowercase{\textit{et al.}}: Bare Demo of IEEEtran.cls for Computer Society Journals}
% The only time the second header will appear is for the odd numbered pages
% after the title page when using the twoside option.
%
% *** Note that you probably will NOT want to include the author's ***
% *** name in the headers of peer review papers.                   ***
% You can use \ifCLASSOPTIONpeerreview for conditional compilation here if
% you desire.



% The publisher's ID mark at the bottom of the page is less important with
% Computer Society journal papers as those publications place the marks
% outside of the main text columns and, therefore, unlike regular IEEE
% journals, the available text space is not reduced by their presence.
% If you want to put a publisher's ID mark on the page you can do it like
% this:
%\IEEEpubid{0000--0000/00\$00.00~\copyright~2007 IEEE}
% or like this to get the Computer Society new two part style.
%\IEEEpubid{\makebox[\columnwidth]{\hfill 0000--0000/00/\$00.00~\copyright~2007 IEEE}%
%\hspace{\columnsep}\makebox[\columnwidth]{Published by the IEEE Computer Society\hfill}}
% Remember, if you use this you must call \IEEEpubidadjcol in the second
% column for its text to clear the IEEEpubid mark (Computer Society jorunal
% papers don't need this extra clearance.)




% for Computer Society papers, we must declare the abstract and index terms
% PRIOR to the title within the \IEEEcompsoctitleabstractindextext IEEEtran
% command as these need to go into the title area created by \maketitle.
\IEEEcompsoctitleabstractindextext{%
\begin{abstract}
%\boldmath
It is common that the software under test(SUT) fails during testing under some specific test configurations(we called failing configurations) and passes under the others. For these failing configurations,in the most general case,not all the options in this configuration is related to the failures. It is desirable to identify the specific options responsible for the failures,i.e., failure-inducing schemas,as by doing this it can reduce the debugging effort.

Combinatorial testing(CT) can effectively detect the errors in SUT which are triggered by interactions of options.However, it provide weak support for identifying them.In this paper, we propose an efficient approach which can aids the CT to locate them.Through using the notions of CMINFS(represent for candidate minimal faulty schemas) and CMAXHS(represent for candidate maximal healthy schemas), our approach can get an efficient performance when identifying the failure-inducing schemas compared to our prior work.Also we make an improvement on the approach to better handle the case where additional failure-inducing schemas may be introduced by newly generated test configurations.

Moreover, we conduct empirical studies on both widely-used real highly-configurable software systems and simulated toy softwares.The results shows that the failure-inducing schemas do exist in software and our approach can identify them more effectively and efficiently than other works.
\end{abstract}
% IEEEtran.cls defaults to using nonbold math in the Abstract.
% This preserves the distinction between vectors and scalars. However,
% if the journal you are submitting to favors bold math in the abstract,
% then you can use LaTeX's standard command \boldmath at the very start
% of the abstract to achieve this. Many IEEE journals frown on math
% in the abstract anyway. In particular, the Computer Society does
% not want either math or citations to appear in the abstract.

% Note that keywords are not normally used for peer review papers.
\begin{keywords}
failure-inducing schemas, fault location, debugging aids, combinatorial testing
\end{keywords}}


% make the title area
\maketitle


% To allow for easy dual compilation without having to reenter the
% abstract/keywords data, the \IEEEcompsoctitleabstractindextext text will
% not be used in maketitle, but will appear (i.e., to be "transported")
% here as \IEEEdisplaynotcompsoctitleabstractindextext when compsoc mode
% is not selected <OR> if conference mode is selected - because compsoc
% conference papers position the abstract like regular (non-compsoc)
% papers do!
\IEEEdisplaynotcompsoctitleabstractindextext
% \IEEEdisplaynotcompsoctitleabstractindextext has no effect when using
% compsoc under a non-conference mode.


% For peer review papers, you can put extra information on the cover
% page as needed:
% \ifCLASSOPTIONpeerreview
% \begin{center} \bfseries EDICS Category: 3-BBND \end{center}
% \fi
%
% For peerreview papers, this IEEEtran command inserts a page break and
% creates the second title. It will be ignored for other modes.
\IEEEpeerreviewmaketitle



\section{Introduction}
% Computer Society journal papers do something a tad strange with the very
% first section heading (almost always called "Introduction"). They place it
% ABOVE the main text! IEEEtran.cls currently does not do this for you.
% However, You can achieve this effect by making LaTeX jump through some
% hoops via something like:
%
%\ifCLASSOPTIONcompsoc
%  \noindent\raisebox{2\baselineskip}[0pt][0pt]%
%  {\parbox{\columnwidth}{\section{Introduction}\label{sec:introduction}%
%  \global\everypar=\everypar}}%
%  \vspace{-1\baselineskip}\vspace{-\parskip}\par
%\else
%  \section{Introduction}\label{sec:introduction}\par
%\fi
%
% Admittedly, this is a hack and may well be fragile, but seems to do the
% trick for me. Note the need to keep any \label that may be used right
% after \section in the above as the hack puts \section within a raised box.



% The very first letter is a 2 line initial drop letter followed
% by the rest of the first word in caps (small caps for compsoc).
%
% form to use if the first word consists of a single letter:
% \IEEEPARstart{A}{demo} file is ....
%
% form to use if you need the single drop letter followed by
% normal text (unknown if ever used by IEEE):
% \IEEEPARstart{A}{}demo file is ....
%
% Some journals put the first two words in caps:
% \IEEEPARstart{T}{his demo} file is ....
%
% Here we have the typical use of a "T" for an initial drop letter
% and "HIS" in caps to complete the first word.
\IEEEPARstart{M}{any} modern softwares are highly configurable and customizable, while this can increase the reusability of  common code base and basic components, make the software easier to be extended, enable the software to run across different environments, however, it also make the testing task to be more challenging. This is because we need to test the software under every possible configurations(we called configuration space) to ensure the correctness of it, which is not feasible in practice. Combinatorial testing technique can handle this problem well. It design a relatively small test suite from the configuration spaces to test the SUT and can ensure to cover all the interactions of option values which not exceed t. When we find some configurations fail in this test suite, which means this suite detect some interaction fault, what we want to know next is to identify which specific interaction options and its values are the cause of these failures. To our best know, combinatorial testing provide week support to identify them.
% You must have at least 2 lines in the paragraph with the drop letter
% (should never be an issue)

 As an example,consider the database MYSQL and some of its configuration options list in Table \ref{table_example}. Assume we generate a test suite consists of available configurations using combinatorial testing technique, and find some test configurations failed, say, (). Then which specific is the cause, is it \textit{extra-charsets=NULL} and \textit{assembler} = \textit{Enabled} , or \textit{extra-charsets} = \textit{NULL} and \textit{assembler=Enabled}  and \textit{innodb-flush-log} = \textit{2} or other possible interactions . Generally, for a failed test configurations consists of n options  have $2^{n} - 1$  possibilities. To identify them is important, for they can help reduce the scope of source code need to be inspected.
\begin{table}\renewcommand{\arraystretch}{1.3}
  \caption{MySQL configuration options example} \centering
  \label{table_example}
  \begin{tabular}{p{0.9\columnwidth}}\hline
  \hline
  \bfseries Compile Time Options\\
   \hline
   \bfseries  Binary Options(Enabled/NULL)\\
    \hline
    assembler, local-infile, thread-safe-client, archive-storage-engine, big-tables, blackhole-storage-engine,client-dflags, csv-storage-engine, example-storage-engine, fast-mutexex, federated-storage-engine, libedit, mysqld-ldflags, ndbcluster, pic, readline, config-ssl, zilb-dir
  \end{tabular}

  \begin{tabular}{c*{2}{p{0.53\columnwidth}}}
  \hline
  \bfseries Non-binary Options &   \bfseries values \\
   \hline
   extra-charsets & -with-extra-charsets=all, -with-extra-charsets=complex, NULL\\
   innodb & -with-innodb, -without-innodb, NULL
  \end{tabular}

  \begin{tabular}{c*{2}{p{0.53\columnwidth}}}\hline
  \hline
  \bfseries Runtime options &   \bfseries values \\
  \hline
   transaction-isolation & READ-UNCOMMITTED, READ-COMMITTED, REPEATABLE-READ, SERIALIZABLE, NULL\\
   innodb-flush-log & 0, 1, 2, NULL\\
   sql-mode & ANSI, TRADITIONAL, STRICT\_ALL\_TABLES\\
  \hline
  \end{tabular}

\end{table}

We call the faulty interactions of option values the failure-inducing schemas, in effect, this notion is not limited in configurable software testing. The input testing may want to find the failure causing input interaction. The GUT testing may want to find the events interaction which makes the software down. The regression testing may want to find which change trigger the new failure. The html testing may want to isolate the minimal faulty related html source code. The Compatibility testing may want to test which some software run simultaneously will crash the system.

How to identify them is not easy, in effect, there are two problems need to be solved: First, we just need to know the properties of the faulty schemas and healthy schemas, i.e., the differences between them. Figuring out that can help us to design approaches to identify a schema to be healthy or faulty. Second, we should find some strategy to check every possible schemas in a test configuration while using as small resources as possible. As the possible schemas in a test configurations can get to $2^n$(n is the number of options in a configuration), then we want the resources needed is logarithmic related to the number of schemas, which will result in the complexity of algorithm with O($n$)(n is the number of options in a configuration).

In this paper, we first give the properties of faulty schemas and healthy schemas. Furthermore, we propose the notion CMINFS(represent for candidate minimal faulty schemas) and CMAXHS(represent for candidate maximal healthy schemas). Through using them we can easily check whether some schemas are healthy or faulty according existed checked schemas. For these schemas can not be checked by exited checked schemas, we generate an newly configuration to test, (do we need to describe  it ?????)and the state pass or fail indicate that whether the schema is healthy or faulty. Note that if newly faulty schemas introduced from the extra test configurations, then the identify result is influenced. We take it into account and solve it by introduce the feedback machinery into our approach.

We studied existed works which also target this problem. To propose a clear view of the characteristics of exited works and our work, we  summary a list of metrics include constraints, limitations, complexities and so on. Through an comprehensive analysis, we have list the results in our paper. Besides these theoretical metrics, we also conduct empirical studies of these approaches. These experiment objects consists of a group of simulated toy softwares, the Small-Scale and Medium-scale Siemens Software Sets and two large highly-configurable softwares: Apache and MySQL. Our results shows that the failure-inducing schemas do existed in software, and our approach can get the best performance when identify them compared with existed works.

\textbf{contributions of this paper}:
1)we study the failure-inducing schema to analysis its properties.
2)we propose a new approach which can identify the failure-inducing schemas effectively.
3)we classified existed works, and give an comprehensive comparison both on theoretical and empirical metrics.

The remainder of this paper is organized as follows: Section 2 introduce some preliminary definitions and propositions. Section 3 describe our approaches for identify failure-inducing schemas. Section 4 give the comparisons in theoretical metrics. Section 5 describe the experiments design. Section 6 list the result of the experiments. Section 7 summarize the related works. Section 8 conclude this paper and discuss the future works.
% An example of a floating figure using the graphicx package.
% Note that \label must occur AFTER (or within) \caption.
% For figures, \caption should occur after the \includegraphics.
% Note that IEEEtran v1.7 and later has special internal code that
% is designed to preserve the operation of \label within \caption
% even when the captionsoff option is in effect. However, because
% of issues like this, it may be the safest practice to put all your
% \label just after \caption rather than within \caption{}.
%
% Reminder: the "draftcls" or "draftclsnofoot", not "draft", class
% option should be used if it is desired that the figures are to be
% displayed while in draft mode.
%
%\begin{figure}[!t]
%\centering
%\includegraphics[width=2.5in]{myfigure}
% where an .eps filename suffix will be assumed under latex,
% and a .pdf suffix will be assumed for pdflatex; or what has been declared
% via \DeclareGraphicsExtensions.
%\caption{Simulation Results}
%\label{fig_sim}
%\end{figure}

% Note that IEEE CS typically puts floats only at the top, even when this
% results in a large percentage of a column being occupied by floats.
% However, the Computer Society has been known to put floats at the bottom.


% An example of a double column floating figure using two subfigures.
% (The subfig.sty package must be loaded for this to work.)
% The subfigure \label commands are set within each subfloat command, the
% \label for the overall figure must come after \caption.
% \hfil must be used as a separator to get equal spacing.
% The subfigure.sty package works much the same way, except \subfigure is
% used instead of \subfloat.
%
%\begin{figure*}[!t]
%\centerline{\subfloat[Case I]\includegraphics[width=2.5in]{subfigcase1}%
%\label{fig_first_case}}
%\hfil
%\subfloat[Case II]{\includegraphics[width=2.5in]{subfigcase2}%
%\label{fig_second_case}}}
%\caption{Simulation results}
%\label{fig_sim}
%\end{figure*}
%
% Note that often IEEE CS papers with subfigures do not employ subfigure
% captions (using the optional argument to \subfloat), but instead will
% reference/describe all of them (a), (b), etc., within the main caption.


% An example of a floating table. Note that, for IEEE style tables, the
% \caption command should come BEFORE the table. Table text will default to
% \footnotesize as IEEE normally uses this smaller font for tables.
% The \label must come after \caption as always.
%
%\begin{table}[!t]
%% increase table row spacing, adjust to taste
%\renewcommand{\arraystretch}{1.3}
% if using array.sty, it might be a good idea to tweak the value of
% \extrarowheight as needed to properly center the text within the cells
%\caption{An Example of a Table}
%\label{table_example}
%\centering
%% Some packages, such as MDW tools, offer better commands for making tables
%% than the plain LaTeX2e tabular which is used here.
%\begin{tabular}{|c||c|}
%\hline
%One & Two\\
%\hline
%Three & Four\\
%\hline
%\end{tabular}
%\end{table}


% Note that IEEE does not put floats in the very first column - or typically
% anywhere on the first page for that matter. Also, in-text middle ("here")
% positioning is not used. Most IEEE journals use top floats exclusively.
% However, Computer Society journals sometimes do use bottom floats - bear
% this in mind when choosing appropriate optional arguments for the
% figure/table environments.
% Note that, LaTeX2e, unlike IEEE journals, places footnotes above bottom
% floats. This can be corrected via the \fnbelowfloat command of the
% stfloats package.
\section{Preliminary}
Before we talk about our approach, we will give some formal definitions and proposals first, which is helpful to understand the background of our description of our approach.

\subsection{definitions}
Assume that the SUT (software under test) is influenced by \emph{n} parameters, and each parameter $c_{i}$ has $a_{i}$ discrete values from the finite set $V_{i}$, i.e., $a_{i}$ = $|V_{i}|$ ($i$ = 1,2,..n). Some of the definitions and propositions below are originally defined in and .

\newtheorem{definition}{Definition}
\begin{definition}[test configuration]
A test configuration of the SUT is an array of \emph{n} values, one for each parameter of the SUT, which is denoted as a \emph{n}-tuple ($v_{1}$, $v_{2}$...$v_{n}$), where $v_{1}\in V_{1}$, $v_{2} \in V_{2}$ ... $v_{n} \in V_{n}$.
\end{definition}
\begin{definition}[test oracle]
the test oracle is that whether it is a test configuration pass or fail. For a clear discuss, we didn't take the multiple output state into account, but we believe our method can easily extend to the multiple oracles.

However, our discuss is based on the SUT is a deterministic software, i.e., will not pass this time and fail that time. We can run our approach multiple times to eliminate this problem, which, however is beyond the scope of this paper.
\end{definition}
\begin{definition}[schema]
For the SUT, the \emph{n}-tuple (-,$v_{n_{1}}$,...,$v_{n_{k}}$,...)is called a \emph{k}-value schema (k > 0) when some k parameters have fixed values and the others can take on their respective allowable values, represented as "-". In effect a test configuration its self is a k-value schema, which k is equal to n. Furthermore, if a test configuration contain a schema, i.e., every fixed value in this schema is also in this test configuration, we say this configuration hit this schema.
\end{definition}
\begin{definition}[subsume relationship]
let $s_{l}$ be a \emph{l}-value schema, $s_{m}$ be an \emph{m}-value schema for the SUT and $l \leq m$. If all the fixed parameter values in $s_{l}$ are also in $s_{m}$, then $s_{m}$ \emph{subsumes} $s_{l}$. In this case we can also say that $s_{l}$ is a \emph{subschema} of $s_{m}$ and $s_{m}$ is a \emph{parent-schema} of $s_{l}$. Additionally, if m = l + 1, then the relationship between $s_{l}$ and $s_{m}$ is direct.
\end{definition}
\begin{definition}[healthy,faulty and pending]
A \emph{k}-value schema is called a \emph{faulty schema} if all the valid test configuration hit the schema trigger a failure. And a \emph{k}-value schema is called a \emph{healthy schema} when we find at least one passed valid test configuration that hits this schema. In addition, if we don't have enough information of a schema,i.e., we don't sure whether it is a healthy schema or faulty schema, we call it the pending schema.

Note that, in real software context, there existed the case that a test configuration hit a faulty schema but passed. We call this case the \emph{coincidental correctness}, which may be caused by other factors which influence the execution result. We will not discuss this case in this paper.
\end{definition}
\begin{definition}[minimal faulty schema]
If a schema is a faulty schema and all its subschemas are healthy schemas, we then call the schema a \emph{minimal faulty schema}(\emph{MINFS} for short).

Note that this is the target to identify,for that this will give us the most precise while enough information to help the developers to inspect the scope of the source code.
\end{definition}
\begin{definition}[maximum healthy schema]
If a schema is a healthy schema and all its parent-schemas are faulty schemas, we then call the schema a \emph{maximum healthy schema}(\emph{MAXHS} for short).
\end{definition}
\begin{definition}[candidate minimal faulty schema]
If a schema is a faulty schema and satisfy the followed condition:
1.none of its subschemas are faulty schemas, 2. at least one subschema is pending schema.(is need discuss!!!!!! one or none is okay?)We then call the schema a \emph{candidate minimal faulty schema}(\emph{CMINFS} for short).
\end{definition}
\begin{definition}[candidate maximum healthy schema]
If a schema is a healthy schema and and satisfy the followed condition:
1.none of its  parent-schemas are healthy schemas, 2. at least one subschema is pending schema. We then call the schema a \emph{candidate maximum healthy schema}(\emph{CMAXHS} for short).
\end{definition}
\subsection{propositions}
\newtheorem{proposition}{Propositions}
\begin{proposition}
All the schemas in a passed test configuration are healthy schemas.
All the subschemas of a healthy schemas are healthy schemas.
\end{proposition}
\begin{proposition}
If schema $s_{a}$ subsumes  schema $s_{b}$ ,schema  $s_{b}$ subsumes schema $s_{c}$, then $s_{a}$  subsumes $s_{c}$.
\end{proposition}
\begin{proposition}
All the parent-schemas of a faulty schema are faulty schemas.
\end{proposition}
\begin{proposition}
All the subschemas of a healthy schema are healthy schemas.
\end{proposition}

\section{The approach to identify the minimal failure-inducing schemas}
Based on these definitions and proportions, we will describe our approach to identify the failure-inducing schemas in the SUT. To give a better description, we will start give an approach with an assumption , and later we will weak the assumption.
\subsection{additional failure-inducing not be introduced}
\newtheorem{assumption}{Assumption}
\begin{assumption}
Any newly generated test configuration will not introduce additional failure-inducing schemas.
\end{assumption}

There are similar assumptions defined in[][], however, it is a strong assumption, which we will change later. And still for this assumption, we can get the followed lemma which can help us to identify whether a pending schema is a healthy schema or a faulty schema.
\newtheorem{lemma}{Lemma}
\begin{lemma}
For a pending schema, we generate an extra test configuration that contains this schema. If the extra test configuration passes, then this schema is a healthy schema. If the extra test configuration fails, then the schema is a faulty schema.
\end{lemma}
\begin{proof}
According to definition of healthy schema, it is obvious that this schema is a healthy schema when the extra test configuration passes.

When the extra configuration fails, this is a faulty schema (or there exists no faulty schema and this test configuration would not fail because the assumption says that this newly generated configuration will not introduce additional faulty schemas).
\end{proof}

\subsection{framework to identify the minimal faulty schema}
As we can identify a pending schema to be healthy schema or faulty schema by generating newly test configurations, then the approach to identify the minimal faulty schema is clear. Fig.\ref{fig_overview} shows an overview of our approach. We next discuss each part of the approach in more detail.
\begin{figure}
 \centering
 \includegraphics[width=2.7in]{gp.eps}
 \caption{overview of approach of identifying MFS}
 \label{fig_overview}
\end{figure}
\subsubsection{Choosing a pending schema}
Before we think getting a pending schema, we should assume an general scenario. That is, we have already make sure some schemas to be healthy schemas and some schemas to be faulty schemas, then assume that we still don't meet the stopping criteria, we will choose a pending to check next. But first, we should make sure what schema is pending schema?

In effect, the schemas we can't make sure whether are faulty schemas nor healthy schemas are our wanted. Step further, we sperate this condition into two parts: 1. can't make sure it is faulty schema. As we already know some faulty schemas, then we just make the schema that first not be any one of these faulty schemas and not be the parent-schema of any one of these faulty schemas(As if not, it must be a faulty schema according to the proposition 3). 2 can't make sure it is healthy schema. Similar to the first condition,we already know some healthy schemas, then we just make the schema that first not be any one of these healthy schemas and not be the subschema of any one of these healthy schemas. It is obvious a pending schema must meet both the two conditions.

However, this is not the end of story of checking a pending schema. With the process of identifying, more and more faulty schemas and healthy schemas will be identified. It is not a small number, as it can reach to O($2^n$). So both for space and time restriction, we should not record all the faulty schemas and heathy schemas. Then we should find another way to check the pending schema.

To eliminate this problem, we propose an method which can check a pending schema with a small cost. It need to record the CMINFS and CMAXHS all through the process. Instead record all the faulty schemas and healthy schemas, CMINFS and CMAXHS are rare in amount, which can help to largely reduce the space need to record. Then according to the followed two propositions, we can easily check a pending schema.

\begin{proposition}
If a schema is neither one of nor the parent-schema of any one of the CMINFS, then we can't make sure whether it is a faulty schema.
\end{proposition}
\begin{proof}
Take a schema $s_a$, a CMINFS set $S_{cminfs}$ and a faulty schema $S_{fs}$ set which determined now. It is note that any element $fs_i \in S_{fs}$ must meet that either $fs_i \in S_{cminfs}$ or $fs_i$ be the parent-schema of one of the $S_{cminfs}$. Assume that $s_a$ is neither the one of nor the parent-schema of any one of the $S_{cminfs}$. To proof the proposition, we just need to proof that $s_a$ is neither the one of nor the parent-schema of any one of the $S_{fs}$.

As $s_a$ is neither the one of nor the parent-schema of any one of the $S_{cminfs}$, so it is not one of $S_{fs}$. Then we assume $s_a$ is the parent-schema of one of $S_{fs}$, say, $fs_j$. As $fs_j$ must meet either $fs_j \in S_{cminfs}$ or $fs_j$ be the parent-schema of one of the $S_{cminfs}$. So $s_a$ is the parent-schema of one of $S_{cminfs}$ according to the proposition 2, which is contradict. So the proposition is correct.
\end{proof}

\begin{proposition}
If a schema is neither one nor the the subschema of any one of the CMAXHS, then we can't make sure whether it is a healthy schema.
\end{proposition}
We ignore this proof as it is very similar to the previous one.

Up to now, we can judge whether a schema is a pending schema, but in effect, there are many pending schemas in a test configurations, especially at the beginning of our process. To choose which one has an impact on our approach. To better illustrate this problem, we consider the followed example:

Assume the failing test configuration: (1 2 1 1 1 2 1 2), that the CMINFS set: (1 2 1 1 - 2 - -) (- 2 - - 1 2 - 2), the CMAXHS set:(- 2 - - - - - -) (- - - - 1 - - -). We list some pending schemas followed (not all, as the number of all the pending schemas is too much that is not suitable to list here):

(1 2 1 - - 2 1 2) (- 2 1 1 - 2 - 2) (1 2 1 1 - - - -) (- 2 1 1 - 2 - -) (1 2 1 - - - - -) (- 2 - 1 - - 2 -) (1 2 - - - - - -) (- - 1 1 - - - -) (1 - - - - - - -)  (- - 1 - - - - -).

Choose what is really different. If we choose (1 2 1 - - - - -), assume we check it as a healthy schema. Then all its subschemas, such as (1 2 - - - - - -) (1 - - - - - - -)  (- - 1 - - - - -), are healthy schemas. It means that we did not need generate newly test configurations for them. But if we check it as a faulty schema, we can make sure all its parent-schemas are faulty schemas, in this case, they are (1 2 1 1 - - - -) and (1 2 1 - - 2 1 2) need no newly test configurations to test.

Let's look at this problem from another angle. Take the schema as a integer number, these parent-schemas of a schema is like the integer numbers bigger then this number, and these subschemas of a schemas is like the numbers smaller than this number. Take a float number as the metric, then, we can describe the faulty schema as the number bigger than this metric, and healthy schema as number smaller than this metric. So the minimal faulty schema is just the number most approximate the metric and bigger than the metric.

It seems like a search problem scenario. Then can we directly apply the efficiently algorithm binary-search? the answer is no, because there are schemas that neither parent-schema or subschema relationship, such as(1 2 1 - - 2 1 2) (- 2 1 1 - 2 - 2). So to utilize the binary search technique, we should make some change. First, should give the followed definitions:

\begin{definition}[chain]
A chain is an ordered sequences of schemas in which every schemas is the direct parent-schema of the schema that follows. Moreover,if all the schemas in a chain are pending schemas, we call the chain a pending chain.
\end{definition}
This definition is similar to the path in []

As all the schemas in a chain are have relationships, then we can apply binary search technique. As we all know, the longer the pending chain, the better performance binary search technique will get. So we should choose a pending chain as longer as possible each iteration.

To get a longest chain, we need to ensure that the head schema of this chain do not have any parent-schema which is a pending schema(called up pending schema), and the tail schema do not't have any subschema which is a pending schema (called down pending schema). The algorithm that get the up pending schemas and down pending schemas are list in algorithm 1 and algorithm 2.

As showed in algorithm 1, we start from the failing test configuration $\mathcal{T}$, assign it to the \emph{rootSchema}(line 1)and then add it to a \emph{lists}(line 3). We then do some operation (line 4 -line 12)to this lists and at last get these pending schemas in lists as up pending schemas.(line 13)  This operation consists of two iteration:

1. successively get one \emph{CMINFS} in $\mathcal{S_{CMINFS}}$. (line 4). Define a temple value \emph{nextLists} which initialize an empty set(line 5).We then execute the second iteration. After that, we will eliminate these same schemas list in \emph{nextLists} (line 10)and then assign to \emph{lists} (line 11).

2. Successively get one schema in \emph{lists}(line 6). And then mutant it according to the \emph{CMINFS} to a set of schemas(line 7). Add them to the  \emph{nextLists} (line 8).

The mutant procedure for a schema is just remove one value in it which this value is also in \emph{CMINFS}. This procedure will result in \emph{k} mutant schemas if the \emph{CMINFS} is a \emph{k}-value schema. By doing this, any mutant schema will not be the parent-schema of the corresponding \emph{CMINFS}.

After this two iteration, the schemas in the \emph{lists} will not be the parent-schema of any \emph{CMINFS} in the $\mathcal{S_{CMINFS}}$.

\begin{algorithm}
  \caption{getting up pending schema}
  \begin{algorithmic}[1]
     \Require  $\mathcal{T}$ \Comment{failing test configuration}

     $\mathcal{S_{CMINFS}}$ \Comment{set of CMINFS}

     $\mathcal{S_{CMAXHS}}$ \Comment{set of CMAXHS}

     \Ensure  $\mathcal{S_{UPS}}$ \Comment{the set of up pending schemas}

    \Statex\Comment{\%comment: initialize\%}
    \State $rootSchema \leftarrow \mathcal{T}$
    \State $lists \leftarrow \emptyset$
    \State $lists \cup \{rootSchema\}$

    \ForAll{$CMINFS$ in $\mathcal{S_{CMINFS}}$}
     \State $nextLists \leftarrow \emptyset$
     \ForAll{$schema$ in $lists$}
        \State$S_{candidate} \leftarrow mutant_{r}(schema,CMINFS)$
        \State $nextLists \leftarrow nextLists \cup S_{candidate}$
     \EndFor
     \State $compress(nextLists)$
     \State $lists \leftarrow nextLists$
    \EndFor
     \State $\mathcal{S_{UPS}} \leftarrow   \{ s | s \in lists \wedge {s\ is\ pending}\}$
  \end{algorithmic}
\end{algorithm}

Fig.\ref{figch} gives an example to the algorithm 1.

\begin{figure}
 \centering
 \includegraphics[width=2.8in]{ch.eps}
 \caption{example of getting up pending schemas}
 \label{figch}
\end{figure}


Algorithm 2 is a bit different from algorithm 1. As our target is to get the minimal value pending schema, we get started from a 0-value schema(line 1). Then to make schemas not to be the subschema of any \emph{CMAXHS} in $\mathcal{S_{CMAXHS}}$, we should let the schemas must contain at least one value that is not in the corresponding \emph{CMAXHS}. So the strategy is to get a reverse schema of the \emph{CMAXHS} which this schema consists of all these values in the failed configuration except these are in \emph{CMAXHS}(line 5). And mutant the schema by adding one value in the reversed schema to make the schma not to be the subschema of the correspinding \emph{CMAXHS}(line 8). After two iteration similar to Algorithm 1, we will get all the schemas that meet that not to be  subschema of any \emph{CMAXHS} in $\mathcal{S_{CMAXHS}}$ from which we choose these pending schemas as down pending schemas(line 14).


\begin{algorithm}
  \caption{getting down pending schema}
  \begin{algorithmic}[1]
     \Require  $\mathcal{T}$ \Comment{failing test configuration}

     $\mathcal{S_{CMINFS}}$ \Comment{set of CMINFS}

     $\mathcal{S_{CMAXHS}}$ \Comment{set of CMAXHS}

     \Ensure  $\mathcal{S_{DOWNS}}$ \Comment{the set of down pending schemas}

    \Statex\Comment{\%comment: initialize\%}
    \State $initschema \leftarrow \mathcal{()}$
    \State $lists \leftarrow \emptyset$
    \State $lists \cup \{initschema\}$

    \ForAll{$CMAXHS$ in $\mathcal{S_{CMAXHS}}$}
     \State $reverse \leftarrow \ reverse(CMAXHS)$
     \State $nextLists \leftarrow \emptyset$
     \ForAll{$schema$ in $lists$}
        \State$S_{candidate} \leftarrow mutant_{a}(schema,reverse)$
        \State $nextLists \leftarrow nextLists \cup S_{candidate}$
     \EndFor
     \State $compress(nextLists)$
      \State $lists \leftarrow nextLists$
    \EndFor
     \State $\mathcal{S_{DOWNS}} \leftarrow   \{ s | s \in lists \wedge {s\ is\ pending}\}$
  \end{algorithmic}
\end{algorithm}

Fig.\ref{figct} gives an example to the algorithm 2. we can see.
\begin{figure}
 \centering
 \includegraphics[width=2.8in]{ct.eps}
 \caption{example of getting down pending schemas}
 \label{figct}
\end{figure}

After we can get the up pending schemas and down pending schemas, then the algorithm of getting the longest chain can be very simple,which is list in Algorithm 3. In this algorithm we can see that we just search through the up pending schemas and down pending schemas(line 7 - 8) to find the two schemas which has the maximum distance(line 9 - 13). The distance of a k-value schema and a l-value schema (k>l) is defined as:

$$distance(S_k, S_l) =
\begin{cases}
-1, & S_k\ is\ not\ parent-schema\ of\ S_l\\
k-l,& otherwise
\end{cases}
$$

Then we just use \emph{makechain} procedure to generate the longest chain. The \emph{makechain} procedure is very simple, it just repeat adding one schema by keeping all the value in down pending schema and removing one factor of the previous schema.

\begin{algorithm}
  \caption{Finding the longest pending schema}
  \begin{algorithmic}[1]
     \Require  $\mathcal{T}$ \Comment{failing test configuration}

     $\mathcal{S_{CMINFS}}$ \Comment{set of CMINFS}

     $\mathcal{S_{CMAXHS}}$ \Comment{set of CMAXHS}

     \Ensure  $\mathcal{CHAIN}$ \Comment{the chain}
    \State $\mathcal{S_{UPS}} \leftarrow getUPS(\mathcal{T},\mathcal{S_{CMINFS}},\mathcal{S_{CMAXHS}}) $
    \State $\mathcal{S_{DOWNS}} \leftarrow getDOWNS(\mathcal{T},\mathcal{S_{CMINFS}},\mathcal{S_{CMAXHS}}) $
    \State $max \leftarrow 0$
    \State $head \leftarrow NULL $
    \State $tail \leftarrow NULL $
    \State $chain \leftarrow NULL $
     \ForAll{$up$ in $S_{UPS}$ }
        \ForAll{$down$ in $S_{DOWNS}$}
            \If {$distance(up, down) \geq max$}
              \State $max \leftarrow distance(up, down)$
              \State $head \leftarrow up$
              \State $tail \leftarrow down$
             \EndIf
        \EndFor
     \EndFor
     \State $\mathcal{CHAIN} \leftarrow  makechain(head, tail)$
  \end{algorithmic}
\end{algorithm}

The last step of getting the schema is just choose the schema from the longest chain. To clearly describe the approach, we discuss it in the overall identifying algorithm which is list in the Algorithm 4. As we have talked previously, we first judge the end criteria, if meet we will report the minimal faulty result. Otherwise we will do the loops. In the loop, we first judge if it is the beginning or headIndex greater than tailIndx, then we will update the CMINFS and CMAXHS, followed generated the longest chain and initial the headIndex, tailIndex and middleIndx. If not we will get the middleIndex indicate the half. Then will select the schema with index of middleindex in the longest chain. And then generate a test configration and exectute the SUT under it. If the result passed, we let tail = middle - 1 else head = middle + 1. By doing this and the previous set the middleIndex we can apply the binary searh to the indentying. It is noted that we intitally let middle = 0. which we want know if this chain has faulty schema as soon as possible.
\begin{algorithm}
  \caption{identify process}
  \begin{algorithmic}[1]
    \Require  $\mathcal{T}$ \Comment{failing test configuration}

     $\mathcal{S_{CMINFS}}$ \Comment{set of CMINFS}

     $\mathcal{S_{CMAXHS}}$ \Comment{set of CMAXHS}

     \While{$hasn't\ meet\ the\ end\ criteria$}
       \If{$the\ beginning$ or $headIndex > tailIndex$}
       \State $update(\mathcal{S_{CMINFS}},\mathcal{S_{CMAXHS}})$
       \State $longeset \leftarrow getLongest( \mathcal{T},\mathcal{S_{CMINFS}},\mathcal{S_{CMAXHS}})$
       \State $headIndex \leftarrow 0$
       \State $tailIndex \leftarrow length(longest) - 1$
       \State $middleIndex \leftarrow 0$
     \Else
       \State $middleIndex \leftarrow \frac{1}{2} \times (tailIndex + headIndex)$
     \EndIf
       \State $\mathcal{SCHEMA} \leftarrow longest[middleIndex]$
       \State $generate\ a\ extra\ test\ configration\ \mathcal{T'}\ contain\ \mathcal{SCHEMA}$
       \State $execute\ SUT\ under\ \mathcal{T'}$
     \If{$the\ test\ configration\ passed$}
       \State $tailIndex \leftarrow middleIndex - 1$
     \Else
       \State $headIndex \leftarrow middleIndex + 1$
     \EndIf
   \EndWhile
   \State $report\ the\ minimal\ faulty\ schemas$
  \end{algorithmic}
\end{algorithm}

\subsubsection{generate a new test configuration}
To generate a new test configuration to test the schema. The generated test configuration must meet the followed rules:

1. must contain the selected schema.

2. must not

3. constraint(system-wide constraint and test case specific constraint, i.e., masking effect)

The first one is easy to meet. We just keep the same value which are in the schema are also in the test configuration.
We must meet the second condition for that if we contain another one, combine this then this test configuration will contain an parent-schema of this selected schema in the original test configuration, and it will confuse us whether it is indicate this schema or its parent-schemas dedicate this result. To fulfil this condition, we need to choose other available values in the SUT which are different from the original test configuration.
the third one is that we should consider the constraints

\subsubsection{execute SUT under the test configuration}
In real software testing scenario, when we test a SUT under a test configuration, there may be many possible testing state: such as pass the testing assertion, don't pass the testing the assertion but with different failure type, can't complete the testing. To get a clear discussion, in this paper we just use \emph{pass} represent the state that pass the testing assertion and \emph{fail} represent all the remained state.

\subsubsection{update information}
The update information is followed when the current chain is checked over. Then before we generate another longest chain, we should update the CMINFS and CMAXHS set. In fact, we just need the CMINFS and CMAXHS in the longest chain.

\subsubsection{stop criteria}
The stop criteria is clear, our algorithm stops when there are no pending schemas left, for that when we once can checked all the schemas of a test configuration, we can get the minimal faulty schemas, which is the target of our algorithm. And whether there are pending schemas can be easily checked by that if we can't generate a longest chain (the length must greater than 1), there must be no pending schemas.
\subsubsection{report the reuslt}
In fact, the last in the CMINFS lists is must be the minimal faulty schemas. For that if these schema in the CMINFS is not minimal faulty schema, then there must be some pending schemas. However, the algorithm stop when there are no pending schemas remained. So at last these in the CMINFS must be the minimal faulty schemas.

\subsection{example}
We will give an complete example followed. Assume that a SUT is . constaints.

\subsection{Without Safe Values Assumption}
feedback machinery


\section{Conclusion}
The conclusion goes here.





% if have a single appendix:
%\appendix[Proof of the Zonklar Equations]
% or
%\appendix  % for no appendix heading
% do not use \section anymore after \appendix, only \section*
% is possibly needed

% use appendices with more than one appendix
% then use \section to start each appendix
% you must declare a \section before using any
% \subsection or using \label (\appendices by itself
% starts a section numbered zero.)
%


\appendices
\section{Proof of the First Zonklar Equation}
Appendix one text goes here.

% you can choose not to have a title for an appendix
% if you want by leaving the argument blank
\section{}
Appendix two text goes here.

The procedure :

$mutant_a$

$mutant_r$

$is pending schema$

$makechain$


% use section* for acknowledgement
\ifCLASSOPTIONcompsoc
  % The Computer Society usually uses the plural form
  \section*{Acknowledgments}
\else
  % regular IEEE prefers the singular form
  \section*{Acknowledgment}
\fi


The authors would like to thank...The authors would like to thank...The authors would like to thank...The authors would like to thank...The authors would like to thank...The authors would like to thank...The authors would like to thank...The authors would like to thank...The authors would like to thank...The authors would like to thank...The authors would like to thank...The authors would like to thank...The authors would like to thank...


% Can use something like this to put references on a page
% by themselves when using endfloat and the captionsoff option.
\ifCLASSOPTIONcaptionsoff
  \newpage
\fi



% trigger a \newpage just before the given reference
% number - used to balance the columns on the last page
% adjust value as needed - may need to be readjusted if
% the document is modified later
%\IEEEtriggeratref{8}
% The "triggered" command can be changed if desired:
%\IEEEtriggercmd{\enlargethispage{-5in}}

% references section

% can use a bibliography generated by BibTeX as a .bbl file
% BibTeX documentation can be easily obtained at:
% http://www.ctan.org/tex-archive/biblio/bibtex/contrib/doc/
% The IEEEtran BibTeX style support page is at:
% http://www.michaelshell.org/tex/ieeetran/bibtex/
%\bibliographystyle{IEEEtran}
% argument is your BibTeX string definitions and bibliography database(s)
%\bibliography{IEEEabrv,../bib/paper}
%
% <OR> manually copy in the resultant .bbl file
% set second argument of \begin to the number of references
% (used to reserve space for the reference number labels box)
\begin{thebibliography}{1}

\bibitem{IEEEhowto:kopka}
%This is an example of a book reference
H. Kopka and P.W. Daly, \emph{A Guide to {\LaTeX}}, third ed. Harlow, U.K.: Addison-Wesley, 1999.


%This is an example of a Transactions article reference
%D.S. Coming and O.G. Staadt, "Velocity-Aligned Discrete Oriented Polytopes for Dynamic Collision Detection," IEEE Trans. Visualization and Computer Graphics, vol.�14,� no.�1,� pp. 1-12,� Jan/Feb� 2008, doi:10.1109/TVCG.2007.70405.

%This is an example of a article from a conference proceeding
%H. Goto, Y. Hasegawa, and M. Tanaka, "Efficient Scheduling Focusing on the Duality of MPL Representation," Proc. IEEE Symp. Computational Intelligence in Scheduling (SCIS '07), pp. 57-64, Apr. 2007, doi:10.1109/SCIS.2007.367670.

%This is an example of a PrePrint reference
%J.M.P. Martinez, R.B. Llavori, M.J.A. Cabo, and T.B. Pedersen, "Integrating Data Warehouses with Web Data: A Survey," IEEE Trans. Knowledge and Data Eng., preprint, 21 Dec. 2007, doi:10.1109/TKDE.2007.190746.

%Again, see the IEEEtrans_HOWTO.pdf for several more bibliographical examples. Also, more style examples
%can be seen at http://www.computer.org/author/style/transref.htm
\end{thebibliography}

% biography section
%
% If you have an EPS/PDF photo (graphicx package needed) extra braces are
% needed around the contents of the optional argument to biography to prevent
% the LaTeX parser from getting confused when it sees the complicated
% \includegraphics command within an optional argument. (You could create
% your own custom macro containing the \includegraphics command to make things
% simpler here.)
%\begin{biography}[{\includegraphics[width=1in,height=1.25in,clip,keepaspectratio]{mshell}}]{Michael Shell}
% or if you just want to reserve a space for a photo:

\begin{IEEEbiography}{Michael Shell}
Biography text here.
\end{IEEEbiography}

% if you will not have a photo at all:
\begin{IEEEbiographynophoto}{John Doe}
Biography text here.Biography text here.Biography text here.Biography text here.Biography text here.Biography text here.Biography text here.Biography text here.Biography text here.Biography text here.Biography text here.Biography text here.Biography text here.Biography text here.Biography text here.Biography text here.Biography text here.Biography text here.Biography text here.Biography text here.Biography text here.Biography text here.Biography text here.Biography text here.Biography text here.Biography text here.Biography text here.Biography text here.Biography text here.Biography text here.Biography text here.Biography text here.
\end{IEEEbiographynophoto}

% insert where needed to balance the two columns on the last page with
% biographies
%\newpage

\begin{IEEEbiographynophoto}{Jane Doe}
Biography text here.Biography text here.Biography text here.Biography text here.Biography text here.Biography text here.Biography text here.Biography text here.Biography text here.Biography text here.Biography text here.Biography text here.Biography text here.Biography text here.Biography text here.Biography text here.Biography text here.Biography text here.Biography text here.Biography text here.Biography text here.Biography text here.Biography text here.Biography text here.Biography text here.Biography text here.Biography text here.Biography text here.
\end{IEEEbiographynophoto}

% You can push biographies down or up by placing
% a \vfill before or after them. The appropriate
% use of \vfill depends on what kind of text is
% on the last page and whether or not the columns
% are being equalized.

%\vfill

% Can be used to pull up biographies so that the bottom of the last one
% is flush with the other column.
%\enlargethispage{-5in}



% that's all folks
\end{document}



